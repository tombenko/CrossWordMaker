\documentclass[10pt,twoside]{book}

\usepackage[paperwidth=148mm,paperheight=210mm,
top=12mm,bottom=12mm,inner=18mm,outer=10mm,
headheight=22pt,headsep=2pt,
dvips]{geometry}

\usepackage{cwpuzzle}

\usepackage[T1]{fontenc}
\usepackage[utf8]{inputenc}
\usepackage[magyar]{babel}

\author{Benkó Tamás}
\title{A CrossWordMaker program leírása}

\begin{document}

\maketitle
\frontmatter
\tableofcontents

\mainmatter

\chapter{A program kezelése}

A program kezelése alapvetően a billentyűzetről történik. Ennek a legjelentősebb oka, hogy a keresztrejtvények túlnyomórészt betűket és számokat tartalmaznak, amiket eleve billentyűzetről viszünk be. Logikus volt tehát úgy elkészíteni a programot, hogy a funkciók szintén elérhetőek legyenek a billentyűzetet használva, így a munka során nem pocsékoljuk a drága időnket az egér és az egérkuzor keresésére és navigálására.

Ellenben a menüket egérrel lehet használni, hiszen ha valamilyen menüparancsot akarunk kiadni, akkor feltehetőleg az aktuális munkát legalábbis félbe tervezzük szakítani, tehát nem lényeges esemény az egér mozgatása.

\section{A képernyő és részei}

Maga a képernyő (a nem feltétlenül létező kereteket leszámítva) két részből áll. Felül egy rövid menüsor található, alatta pedig maga a kitöltésre váró rejtvényábra.

A menüsor három menüpontot tartalmaz, ezek sorrendben:
\begin{itemize}
\item File
\item Beállítások
\item Információk.
\end{itemize}
Az itt található menüpontokról később még lesz szó.

A rejtvényábrában halványszürke négyzet viselkedik kurzorként, ez mutatja, hogy a következőnek beírt betű illetve szám hová kerül.

Az ablakot átméretezni nem lehet, csak a megjelenített ábra méretének megváltoztatásával.

\section{Szerkesztés}

A szerkesztés a billentyűzet segítségével történik. Alapvetően amikor egy billentyűt lenyomunk, akkor a hozzá tartozó betű (ha van ilyen) az éppen kijelölt négyzetbe kerül, majd a kurzor a következő mezőre ugrik.

Számbillentyűk lenyomásakor a számjegy a négyzetben található szám végéhez hozzáfűződik. A program nem ellenőrzi a számok hosszúságát, normális esetben ugyanis három jegynél hosszabb szám nem kerülhet a négyzetbe. (Persze mindig van nem normális eset is, de az így járt.)

A kurzor mozgási iránya alpértelmezésben balról jobbfelé mutat, de ezt a szóköz (space) billentyűvel átválthatjuk fentről lefelé irányra. Visszafelé ugyanígy kell cselekedni.

A törlés (delete) billentyű hatására az aktuális négyzet teljes tartalma törlődik, beleértve a számokat is. Ezután a kurzor a következő négyzetre lép. A visszalépés (backspace) billentyűnél is ez történik, csak a lépés ellentétes irányba történik.

A rejtvényekben található egyéb jelöléseket szintén a billentyűzetről lehet bevinni. A pont lenyomása fekete négyzetet jelent, ezt a program betűként értelmezi, tehát törölni a törlőgombokkal lehet. A vastag vonalakat a pont mellett található két gombbal lehet ki és bekapcsolni. A mínuszjel a függőleges, a vessző a vízszintes vonalakat határozza meg.

A kurzor független mozgatására a kurzornyilak szolgálnak, itt értelemszerűen történik a mozgás.

\section{A menük}

\subsection{File menü}

Ebben a menüben a különféle ki és bemeneti műveleteket végezhetjük el. A menüpontok a következők:

\subsubsection{Új} Új keresztrejtvény szerkesztéséhez láthatunk hozzá. Első lépésként megjelenik egy ablak, ahol beállíthatjuk a rejtvény dimenzióit.

\subsubsection{Mentés} A meglévő rejtvényt fájlba írhatjuk.

\subsubsection{Visszaállítás} Egy korábban elmentett rejtvény szerkesztését folytathatjuk.

\subsubsection{Exportálás LaTeX-be} A rejtvényt a nagytudású nyomdai szedőprogram, a \LaTeX{} \emph{cwpuzzle} makrócsomagjának formátumában lehet elmenteni. Ennek előnye, hogy a rejtvényábra nyomdakész állapotba kerül, illetve meghatározásokat és egyéb finomságokat lehet hozzáadni.

\subsubsection{Nyomtatás} Az ábra kinyomtatását végezhetjük el.

\subsubsection{Kilépés} A program használatát fejezhetjük ezzel a menüponttal be.

\subsection{Beállítások menü}

\subsection{Információk menü}

\chapter{A keresztrejtvényekről}

\section{Történelem}

\section{Alaptípusok}

\subsection{Hagyományos}

\subsection{Vonalas}

\subsection{Olasz}

\subsection{Skandináv}

\subsection{Kocka és vonalvadászat}

\subsection{Kevert}

\subsection{Újraszerkesztés}

\subsection{Keretjáték}

\section{Speciális rejtvénytípusok}

\subsection{Plusz egy vicc}

\subsection{Karácsonyfa}

\subsection{Ugrás a sötétbe}

\chapter{Ajánlások}

\chapter{A program belső felépítése}

\backmatter

\chapter{Függelék: A GNU Public Licensz}

\end{document}